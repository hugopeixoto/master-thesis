\chapter{Introduction} \label{chap:intro}

\section*{}


\section{Context and Motivation}
\label{section:context}

Municipal solid waste (MSW) production has been increasing in the last few
years, along with economic growth\citep{McCarthy94}. This has led to the need
--- and subsequent development --- of efficient waste management solutions.
Waste management involves not only the collection, but also the transportation,
recycling and disposal of generated waste.

One can find several case studies reporting the methodologies used by different
countries, as well as reports describing the MSW generation quantities and
patterns.

According to \citet{Bhat1996}, up to 85\% of some cities' waste management
budget spent goes to MSW collection and disposal. With this in mind, route
optimization can be considered as an important field of study regarding the
improvement of MSW management processes.

Many cities around the world have studied and applied several optimization
techniques to their collection scenarios, which can be very different from each
other. In Portugal, however, there seems to be little research regarding this
subject. \citet{Passaro200397} describes the waste management scenario in
Portugal from 1996 and 2002, a period in which several improvements were made
due to changes in the legislation. In 2006, \citet{Magrinho20061477} further
report the legislation trends and present some statistics on the average MSW
generation rate. Concerning collection routing, \citet{Teixeira04} describes a
study for optimizing the collection of urban recyclable waste in the
centre-littoral region of Portugal.




\section{Fill Status Monitoring}
\label{section:monitoring}

The Municipality of Porto, Portugal, is working together with the Fraunhofer
Portugal Research Center for Assistive Information and Communication Solutions
(FhP-AICOS) in order to implement a platform that allows the real-time
measurement of waste containers' fill status. This requires the deployment of
low cost sensors in each container and the development of a communication
system to gather information. On top of this platform, several applications
will then be possible. As an example, this will allow researchers to study
waste generation behaviours on a more detailed level. These monitoring systems
have been studied in places, such as Pudong New Area, in Shanghai,
China\citep{Rovetta09,Vicentini09} and Sweden\citep{Johansson06}.

Another application for this system is the optimisation of waste collection
routes.




\section{Optimisation of Waste Collection Routes}
\label{section:optimization}

The problem of optimising waste collection routes involves deciding, for
example, which streets must each garbage truck follow, which containers should
each one of them collect and how many trucks should a fleet for a given city
have.

One of the first articles regarding this subject was done in
1974\citep{Beltrami74}, and it was applied to both New York and Washington
D.C., United States of America. Since then, several other cities have tried to
minimise the costs involving waste collection: Trabon,
Turkey\citep{Apaydin2007}; Barcelona, Spain\citep{Bautista2004}; Athens,
Greece\citep{Karadimas2005}; Hanoi, Vietnam\citep{Tung2000}; Porto Alegre,
Brazil\citep{Li2008} and many others. 

However, many of these studies do not have real-time information of the
containers' fill status. Usually, they are either based on statistical data
(surveys), or they ignore the containers' fill status and simply collect the
waste in every container.

Combining these techniques with the \textit{Fill Status Monitoring} platform
described in section~\ref{section:monitoring}, the municipality of Porto,
Portugal might reduce even further the collection costs. The main objective of
this work is to develop a solution for route optimisation based on the
previously presented platform.



\section{Objectives}
\label{section:objectives}

With the deployment of a fill status monitoring solution in the municipality
of Porto, there is the need to develop an optimisation framework for the
waste collection routes.

There are two scenarios that will need optimisation. First, there is a
scenario similar to the \textit{Rollon-Rolloff} scenario. Second, a
\textit{commercial} scenario will also be evaluated. The difference between
these can be seen in Chapter~\ref{chap:sota}.

Other than the optimisation process, there is also the need to develop an
architecture to store and retrieve, when necessary, the information obtained
from the fill sensors.


\section{Organisation}

Chapter~\ref{chap:sota} starts by presenting the optimisation of collection routes.
First, an informal description of each scenario is given. Then, each one of the
scenarios is exposed as a mathematical formulation, followed by possible
techniques that can be applied to them.

Chapter~\ref{chap:approach} introduces the optimisation framework being
developed. Three modules are proposed and detailed, and the progress on each
one is presented.

Chapter~\ref{chap:work} presents a work plan for the project.  The work plan
contains both the specification of the tasks, as well as a calendarisation.

