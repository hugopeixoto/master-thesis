\chapter{Conclusions}
\label{chap:conclusions}

\section*{}

This final chapter presents a review of the relevant information obtained from
this study and an exposition of further research that could be done in this
area. Section~\ref{section:conclusions} presents the sensibility analysis
results regarding the \textit{MAX-MIN ant system} and describes results on the
application of the \textit{route-first-cluster-second} approach to the waste
collection scenario.

Section~\ref{section:future-work} exposes some thoughts on how to further extend
the study on both the \textit{Asymmetric Traveling Salesman Problem} and the
\textit{Capacitated Vehicle Routing Problem}. It also finalizes this document by
presenting the next stages of implementing the waste collectionwaste collection
optimization framework.


\section{Conclusions}
\label{section:conclusions}

This document presented an architecture proposal for the storage and retrieval
of containers' status, together with formats for information interchange. These
formats are built on top of JSON, an open standard that can be used to represent
simple data structures. An alternative would be to use XML; however, JSON was
preferred over XML due to the following reasons:

\begin{itemize}
  \item JSON has a simpler notation than that of XML;
  \item JSON was designed to serve as a data interchange format, whereas XML
  was designed to be a document interchange format.
\end{itemize}

The second goal of this work is to provide insight on existing algorithms to
calculate efficient routes for waste collection. This problem is modeled as an
\textit{Asymmetric Capacitated Vehicle Routing Problem} (ACVRP). Several methods
have been proposed, in the literature, on how to approach the ACVRP.

To solve the ACVRP, there are construction heuristics (such as the one by
Clarke-Wright). These yield a set of vehicle routes, which can be further
optimized by applying ATSP solving techniques.

Another method to address the ACVRP involves solving the associated
\textit{Asymmetric Traveling Salesman Problem} (ATSP) and dividing the resulting
tour into routes that respect the vehicle capacity constraints. This is called a
\textit{route-first-cluster-second} approach.

To solve the ATSP, several techniques were compared: two construction
heuristics --- \textit{greedy} and \textit{repetitive nearest neighbor} (RNN)
--- and three meta-heuristics --- \textit{hill climbing} (2-opt),
\textit{genetic algorithms} (GA) and \textit{MAX-MIN ant system} (MMAS).
Apart from their comparison regarding the efficiency of resulting routes,
MMAS was also targeted for a parameter sensibility analysis.

\subsection{Sensitivity analysis of MMAS}

As a result of the parameter sensitivity analysis for the \textit{MAX-MIN ant
system}, it was shown that, for the given datasets, the optimum value for
parameter $\beta$ lies in the interval $[5, 20]$. Although the optimum value
appears to be located closed to the lower side of this interval, it is
recommended that higher values are used. The rationale for this is that there is
a significantly steep ascent for values close to $\beta=5$. Thus, it was chosen
to use $\beta=20$ during this study. This helps to prevent against possible
oscillations in the $\beta$ optimization curve, when applying MMAS to larger
graphs.

Regarding the number of ants and iterations, it was shown that increasing both
parameters increases the resulting route efficiency. When applying this
algorithm with a limited time frame, though, it is preferable to have a low
number of ants and a high number of iterations. However, one must take into
consideration that using MMAS with a low number of ants may lead to convergence
too soon. Early convergence causes the algorithm not to take full advantage of
the total available running time.

\subsection{Comparison of methods for the ACVRP}

When comparing the five methods to solve the ATSP, MMAS obtained more efficient
routes than the other four methods. In average, the improvement of MMAS over
the best route obtained using the other methods is of $2.6\%$.

Although MMAS outperforms the other techniques in solving the ATSP, the same
needs not to hold for the ACVRP solutions. Immediately after clustering, the
hill climbing approach applied to the \textit{greedy} heuristic produces more
efficient routes for half of the fifteen large datasets. Nevertheless, MMAS
only exceeds the best solution for each dataset, in average, by $2.2\%$.

The third step of the optimization consists of applying an ATSP technique to
each of the vehicle routes obtained from the previous step. After this step,
MMAS is still the approach that yields better results, with an average excess of
$2.6\%$. This technique also has the lowest standard deviation ($\sigma=3.4\%$),
which means that it is the most stable algorithm.

In terms of solution efficency, MMAS is followed by the \textit{greedy}
heuristic, which has an average excess of $3.2\%$ ($\sigma=4.2\%$). This shows
that the initial ATSP solution performance does not directly influence the
efficiency of the final ACVRP set of routes.

It was also shown that the \textit{savings} heuristic performs better when
followed by a MMAS optimization, rather than the 2-opt technique. This is
consistent with the fact that MMAS had performed better than 2-opt when applied
to ATSP instances. This leads to the conclusion that it would be possible to
further reduce the route costs by applying MMAS at the final optimization step,
instead of 2-opt.

A 12-page article summarizing the results of this study was submitted to an
international workshop on algorithmic approaches for transportation modeling.
The preliminary version can be seen in appendix~\ref{appendix:article}.


\section{Future work}
\label{section:future-work}

Regarding route optimization, this study analyzed a constructive heuristic and
several \textit{route-first-cluster-second} approaches for solving the
\textit{Capacitated Vehicle Routing Problem}. It would be possible to apply
\textit{cluster-first-route-second} methods to these approaches, although most
of them rely on predefining a fixed number of vehicles to use.

It would also be interesting to apply MMAS and other techniques at the last
optimization step, instead of simply using $2-opt$. This task was not done due
to the time it would take to run all benchmarks.

Having designed the architecture for information management regarding waste
containers' fill status, the next step is to implement a web service
application layer between the central server and the sensors. This would
consist of sending the container's fill status using through an HTTP request.
Depending on the sensors' network architecture --- currently being studied in
Fraunhofer Portugal Research Center --- authorization policies and mechanisms
might have to be designed.


